\section{introduction}
\subsection{背景}
近年來數據以飛快的速度成長,TB或是PB等級的數據隨處可見,在這資料快速產生且數據快速交換的時代,大數據一詞也越常被提及,國際數據公司(International Data Corporation, IDC)有研究指出,2008年全球生產的資料量為0.49ZB,2009年全球產生的資料量為0.8ZB,2010年增長為1.2ZB,2011年的資料量更是已經達到1.82ZB,這相當於全球每人生產200GB的資料,這麼龐大的資料也成了產業界與學術界所需要探討的重要議題,而有這麼大量的資料也意味著會有大量的應用會產生,而這一些應用當中一定會需要資料的交換,而在教會資料的時候,大多數的應用會選擇XML。\\\par
在大數據中,有所謂的5V,所謂的5V是指Volume, Value, Veracity, Vleocity, Variety,Volume是指產生的資料量
