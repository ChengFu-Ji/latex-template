% !TEX encoding = UTF-8 Unicode
\documentclass[a4paper]{article}
\usepackage[margin=3cm]{geometry}%邊界
\usepackage{xeCJK, fontspec, type1cm, txfonts, indentfirst, cite}
\usepackage{subcaption}
\usepackage{booktabs}
\usepackage{listings}%程式碼
\usepackage{graphicx}%載入圖片
\usepackage{caption}
\usepackage{algorithm}
\usepackage{algpseudocode}
\usepackage{amsmath}
\usepackage{url}
\usepackage{wallpaper}% water mark
\bibliographystyle{unsrt}
\setmainfont{Times New Roman}
\XeTeXlinebreaklocale "zh"
\setCJKfamilyfont{xxm}{PMingLiU}
\CenterWallPaper{.18}{image/watermark.jpg}
\title{基於Apache Spark 建構串流XML Veracity 真實度之模型}
\date{}

\begin{document}
\maketitle
\renewcommand{\abstractname}{中文摘要}
\begin{abstract}
近幾年大數據的數據量飛快地成長,超過TB等級的數據已是隨處可見,而數據傳輸在大數據中是一個重點探討的問題,大數據在做資料交會的時候,由於資料量龐大,所以資料的真實度不易掌控,也就是大數據5V當中的Veracity是我們最關切的問題。\\\par
而XML(延伸標記語言 eXtensible Markup Language)作為現今通用的網路資料交換格式,隨著網際網路資料的增長,也已經同樣具有大數據(Big Data)的特徵,在近幾年來,產業界與學界都將大數據列為重要研究議題,並投入相當多的資源支持大數據的研究。\\\par
本研究提出使用Apache Spark建構串流XML Veracity真實度模型以及真實度模型應用程式開發介面(Veracity Model Application Programming Interface, Veracity Model API),來解決大數據在資料傳輸當中我們所關心的真實度Veracity的問題,並可以讓使用者知道文件有多少可信度,XML真實度模型計分是一個給定一個被量測XML文件,讓模型進行被量測XML文件與標準XML文件的比較,並給定一個量化分數的系統,本研究建構一個利用Apache Spark 的平行化處理,加速真實度模型整體的運算速遞及效能的模組,讓使用者進行串流的XML文件上傳,再將上傳文件放入由Apache Spark建構的真實度計分模型來進行真實度計分,XML文件真實度有很多面向可以做詮釋,本研究設計一個基於物件導向程式設計的真實度模型,藉由物件導向的特性,設計抽象類別來建構一個真實度的基本框架$(Dimension\ i, D_i)$,再讓使用者繼承此類別實作模型的功能,換言之就是讓使用者決定需要有什麼樣的屬性$(Attribute\ j, P_{i, j})$才可以評價真實度,以及需要有什麼樣的量化和計分方式才可以將真實度呈現出來,如此這個模型的設計將具有彈性且有助於使用者在面對大量XML資料的時候能狗有一個真實度計分的量化標準,在未來開發大數據應用的時候,可以對於在傳輸資料的時候有一個依據來知道資料的可靠程度。
\end{abstract}
\newpage
\tableofcontents
\newpage
\listoffigures
\newpage
\section{introduction}
\subsection{背景}
近年來數據以飛快的速度成長,TB或是PB等級的數據隨處可見,在這資料快速產生且數據快速交換的時代,大數據一詞也越常被提及,國際數據公司(International Data Corporation, IDC)有研究指出,2008年全球生產的資料量為0.49ZB,2009年全球產生的資料量為0.8ZB,2010年增長為1.2ZB,2011年的資料量更是已經達到1.82ZB,這相當於全球每人生產200GB的資料,這麼龐大的資料也成了產業界與學術界所需要探討的重要議題,而有這麼大量的資料也意味著會有大量的應用會產生,而這一些應用當中一定會需要資料的交換,而在教會資料的時候,大多數的應用會選擇XML。\\\par
在大數據中,有所謂的5V,所謂的5V是指Volume, Value, Veracity, Vleocity, Variety,Volume是指產生的資料量

\newpage
\section{相關研究}
本研究建構一個在Apache Spark 叢集上的Veracity真實度模型,且應用於串流XML的真實度計算,而以往的真實度研究在大部分的文獻中都是討論文件相似度。再來XML具有樹狀結構以及自描述(self-descriptive)特性,所以在分散式系統處理XML的時候,如何切割文件,但依然保有XML文件樹狀結構和父子節點的關係,以及在Hadoop或是Apache Spark的分散式架構下做XML的Query。本章節將就有關這些議題的文獻來做討論。
\subsection{XML文件特性}
XML結構與網頁使用的HTML十分相似,而兩者最大的不同為XML的設計為用來做資料傳輸,而HTML的設計是用來呈現資料。再者XML雖然可以自行定義Tag名稱,但有嚴格的巢狀結構規定。而HTML雖然在Tag的命名上沒有那麼自由,但有一些Tag沒有遵守巢狀結構的規定卻依然可以作動。\\\par
在\cite{w3sxml}\cite{2005xml}當中,針對XML的特性描述提到,XML可以表示成樹狀結構以及XML具有自描述(self-descriptive)的特性,以及XML可以使用如DTD或是Schema來規範其內容結構。\\\par
在自描述性中,XML可以使用標籤描述資料內容,如圖\ref{self}所示,文件裡面的Tag具有描述資料的功能,XML不負責呈現資料,所以程式開發者需要另外撰寫程式來完成。

\begin{figure}[H]
\centering
\graphicspath{{/Users/FUDA/Documents/masterThesis/image/}}
\includegraphics[scale=0.45]{xmlself.png}
\caption{XML 自描述範例}
\label{self}
\end{figure}
而XML的缺點在於需要用Tag來存放資料,以文字檔來說容量會比較大,但以現今的網路傳輸速度以及資料壓縮的技術,這也不再是問題。
\subsection{XML文件的平行化處理}
Hongjie et al.的文獻\cite{fan2018handling}提到將大型XML切割成小型的樹狀結構,存進分散式檔案系統,等到使用者的query進來之後,再將切割好的資料提取出來,使用MapReduce進行查詢。利用Hadoop的分散式系統,來到平行化query。文獻當中使用分散式檔案系統儲存XML 文件,也就是說文件在儲存之前需要經過切割。文章當中他們是採用自己設計的切割演算法,將大型XML 文件的樹狀結構切割成小型樹狀結構,接著在使用者的query進入的時候,會平行化的對這一些切割出來的小型樹狀結構作查詢。\\\par
這裡面有幾個問題點,第一點是當有多個XML文件的時候,每一份文件都會切割演算法做切割並且儲存,這時候會產生很多小文件,如何對應切割出來的小文件與原始大文件的關係,這會影響到要對哪一個文件進行處理,而文件切割和對應的部分,在\cite{eiffcientXML}當中有提到,一般在Hadoop當中,透過MapReduce切割XML文件的時候,開始標籤(<tag>)與結束標籤(</tag>)之間的關係會被切割到不同的部分,導致XML文件的節點關係不清楚,第二點是文件切割與平行化的問題,在\cite{fan2018handling}當中是使用Hadoop做處理,Hadoop 可以自行決定任務的平行化數量。而如何計算和得知切割XML的個數與平行化任務的數量各為多少是比較好的,這是在做平行化運算要面臨的問題點。\\\par

可以看到圖\ref{xmlhadoop}中紅框標示的部分即為XML文件切割完之後要進行平行化運算的部分。當中我們比較關注的是XML文件怎麼切割?被切割了幾份?以及平行畫的時候會產生的任務數量以及計算量,都是我們要考量的問題。

\begin{figure}[H]
\centering
\graphicspath{{/Users/FUDA/Documents/masterThesis/image/}}
\includegraphics[scale=0.4]{xmlHadoop.png}
\caption{XML在Hadoop的切割與平行化}
\label{xmlhadoop}
\end{figure}

\subsection{XML文件相似度}
在本研究所提出的Veraicty之問題在以往的相關研究當中算少數,大部分探討是以XML文件相似度為主。在以往的研究當中會使用樹的編輯距離\cite{bille2005survey}(Edit Distance,或稱Levenshtein Distance)作相似度的比較,所謂的編輯距離為給定XML文件A與XML文件B,如果要使文件B變成跟文件A相同,需要做多少次新增、刪除以及修改的動作。在\cite{tai1979tree}當中即是採用此方法驗證兩份文件的相似度。\\\par
而在\cite{tekli2006semantic}以及\cite{tekli2012novel}當中為了增進效能,使用的是子樹(sub-tree)來做相似度的計算。而兩篇文獻不同的點在於\cite{tekli2006semantic}使用的是編輯距離,而\cite{tekli2012novel}是對於子樹的結構以及語意相似的子樹出現次數或重複次數進行計算。\cite{tekli2015approximate}則提出了基於樹編輯距離的XML語法(XML grammar)相似度驗證,並且一樣使用編輯距離來比較XML以及XML語法的相似度。\\\par
接者在\cite{algergawy2010element}當中則採用了多種的驗證方式,如Tag 名稱相似度、編輯距離、N-grams距離、Tag語意驗證等。而對於編輯距離的驗證方式除了常見的Levenshtein Distance還有使用Jaro Similarity的相似度驗證。Jaro 為兩個字串的相似度的度量,如果Jaro值越高,代表相似度越高。而另一個則為N-gram相似度,作法為將兩組欲比較之字串按照長度N切割,則可以計算兩個字串相同的子字串有多少,進而比較原字串相似度。\\\par
上述的相關研究大部分是使用編輯距離相似度演算法來做XML相似度的問題。而考量到計算效能上的問題,有幾篇的研究採用了子樹的計算來降低計算量進而強化效能。然而這樣的相似度比較對於真實度來說只是其中一個維度。而且無論是哪一種方法,在現今龐大的資料量以及串流數據的場景,皆很難完成即時性的處理。

\newpage
\newpage
\newpage
\section{Veracity真實度模型}
現今大數據的興起,資料科學也是風行全球。然而在做資料分析的時候,大家最關心的是資料的可性度,也就是大數據5V當中的Veracity。而建構在資料理解性上的真實度時,因為每一個人在看資料的時候,所看的面向是不同的,所以這就造成的標準不同。所以本研究提出Veracity 真實度模型來提出一個量化的方式解決這個問題。
\subsection{模型理論}
真實度模型是建構在資料理解性(data understandability)上,而所謂的資料理解性不像一般常見的對於XML文件相似度的比較。例如要說這兩份資料的真實度很高,也許大家從不同角度觀察,結果都不盡相同。這樣評量資料真實度就顯得很抽象,本研究即提供一個量化的方式,也就是真實度模型來針對XML做分數量化,讓使用者自行設計量化方法,來決定如何把抽象化為具體的評比分數。\\\par

而在真實度模型(Veracity Model)的架構之下會有多個維度(Dimension),在維度之下會有的多個屬性(Attribute ),每一個屬性之下都有其量化的方法(Quantification)。假設有一份基準XML文件B(Base XML Document)以及被量測文件M(Measure XML Document),而文件M將被量測與文件B的真實程度(Veracity Value)。這樣就可以計算出文件M在真實度模型中的真實分數。而每一個維度也可以根據下面所屬的屬性計算出維度的分數(Dimension Degree),也可以得到維度下方每一個屬性的分數(Score)。\\\par

真實度模型($M$)的架構當中,會有多個真實維度來表示使用者理解資料的多個面向。每一個維度會描述使用者對於資料理解性的每一個面向。下面定義了一個模型有n個維度,$D_i$表示模型當中的第n個維度:
$$
M=\{D_i\}\ ,\ i= 1\ to\ n
$$
在真實維度$D_i$當中會有多個真實屬性來描述真實維度的細節。舉例來說,使用者認為資料來源可視為一個維度,那麼這個維度的屬性可能會有來源網址、文件編碼等等,這樣可以定義成真實維度$D_i$當中的$A_{i,j}$個屬性:
$$
D_i=\{A_{i,j}\}\ ,\ j=1\ to\ n
$$
在真實屬性$A_{i,j}$當中,會有其量化的方法$Q_{i,j}$。讓使用者可以針對每一項屬性去描述如何量化以及在量化的時候可能需要做的正規化。\\\par

綜合以上,我們可以從每一個維度的真實屬性中的量化方式得到屬性的分數。且根據不同屬性做權重$w_{i,j}$的調整,再自定義函數$F_A()$來計算所有屬性與權重,進而得到維度的分數:
$$
\left\lvert D_{i}\right\rvert=F_A(w_{i,j},\  \left\lvert Q_{i, j}\right\lvert)
$$
%整個真實度模型的估算方式Fd
而從上述所得到的維度分數與權重$W_{i}$做調整,並且放入自定義函數$F_D()$中即可得到整體真實度模型的分數:
$$
 \left\lvert M\right\lvert=F_D(W_i,\ \left\lvert D_{i}\right\lvert)
$$

\subsection{UML表達方式}
基於上述理論模型,使用UML建構模型之物件導向關係圖。真實度模型使用UML來表示如圖\ref{uml}所示。在圖\ref{uml}當中有Model、Dimension和 Attribute這三個抽象類別。在設計的時候,每一個類別都必須繼承這三個類別的其中一個,並且實作內容。實做出來的子類別則會產生組合的關係。如Model下面會有多個Dimension;Dimension下則會有多個Attribute,且三個類別互相相依,當Dimension下沒有Attribute的時候,這個Dimension將不存在;而當Model 下沒有Dimension,則這個Model也不存在。

\begin{figure}[H]
\centering
\graphicspath{{/Users/FUDA/Documents/masterThesis/image/}}
\includegraphics[scale=0.42]{uml.png}
\caption{真實度模型}
\label{uml}
\end{figure}

上述之UML提供一個物件導向的程式架構,把前面的理論模型轉換為可程式化的方法。也就是只要有這樣的模型以及敘述,使用任一種程式語言皆可以實作本研究之模型,進而達到使用者在選擇程式語言實現本模型時候的彈性。\\\par

在實現本模型的時候,需要注意的是抽象類別的繼承與改寫以及物件的組合關係。首先是抽象類別的繼承與改寫,使用者須繼承抽象類別並進行實作。假設進行了繼承卻沒有實作,那麼程式將無法正常動作。再來是物件的組合關係,前面說到三個類別是具有相依關係的。也就是在實作的時候,每一個Dimension下至少要有一個Attribute,這樣這個Dimension才得以存在。而每一個Model下至少要有一個Dimension,這樣Model才算成立。
\newpage
\section{Veracity模型API}
本研究使用OOP的觀念與特性在Apache Spark上面實做真實度模型,在前面有提到對於資料的真實度評價可以從很多面向來做探討,換句話說,可以讓使用者決定所謂的真實度需要包含哪一些屬性,還有真實度的量化方法、量化公式等,藉由OOP的方式,可以讓Veracity真實度模型的詮釋方式更有彈性,量化方式可以隨著不同的應用做變化,搭配Apache Spark 可以加速整個對於大型XML文件處理的效能。
\subsection{物件導向程式設計}
物件導向程式設計(Object-Oriented Programming,簡稱OOP)是一種抽象化的程式開發方法,
\subsection{模型建立}
Veracity真實度模型的建立,其中最重要的就是彈性,也就是用物件到像程式設計的特性來達到
\subsection{模型實作}
\end{document}
