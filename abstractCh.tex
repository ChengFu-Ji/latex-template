\addcontentsline{toc}{section}{摘要}
\begin{center}

\large{基於Apache Spark 建構串流 XML Veracity 真實度之模型}\\
\vspace{0.83cm}

\makebox[3cm][r]{\large{學生:}}
\makebox[3cm][l]{\large{紀承甫}}
\hfill
\makebox[3cm][r]{\large{指導教授:}}
\makebox[3cm][l]{\large{陳世穎}}
\vspace{0.83cm}
\\
\large{國立臺中科技大學}\large{ 資訊工程系}
\vspace{0.83cm}
\renewcommand{\abstractname}{摘要}
\begin{abstract}
%近幾年大數據的數據量飛快地成長,超過TB等級的數據已是隨處可見。而數據傳輸在大數據中是一個重點探討的問題,大數據在做資料交換的時候,由於資料量龐大,所以資料的真實度不易掌控,也就是大數據5V當中的Veracity是我們最關切的問題。\\\par
%而XML(延伸標記語言 eXtensible Markup Language)作為現今通用的網路資料交換格式,隨著網際網路資料的增長,也已經同樣具有大數據(Big Data)的特徵。在近幾年來,產業界與學界都將大數據列為重要研究議題,並投入相當多的資源支持大數據的研究。\\\par
%本研究提出使用Apache Spark建構串流XML Veracity真實度模型以及真實度模型應用程式開發介面(Veracity Model Application Programming Interface, Veracity Model API),來解決大數據在資料傳輸當中我們所關心的真實度Veracity的問題,並可以讓使用者知道文件有多少可信度。XML真實度模型計分是一個給定一個被量測XML文件,讓模型進行被量測XML文件與標準XML文件的比較,並給定一個量化分數的系統。本研究建構一個利用Apache Spark 的平行化處理,加速真實度模型整體的運算速遞及效能的模組。讓使用者進行串流的XML文件上傳,再將上傳文件放入由Apache Spark建構的真實度計分模型來進行真實度計分,XML文件真實度有很多面向可以做詮釋。本研究設計一個基於物件導向程式設計的真實度模型,藉由物件導向的特性,設計抽象類別來建構一個真實度的維度$(Dimension\ i, D_i)$,再讓使用者繼承此類別實作模型的功能。換言之就是讓使用者決定需要有什麼樣的屬性$(Attribute\ j, A_{i, j})$才可以評價真實度,以及需要有什麼樣的量化和計分方式才可以將真實度呈現出來。如此這個模型的設計將具有彈性且有助於使用者在面對大量XML資料的時候能夠有一個真實度計分的量化標準。在未來開發大數據應用的時候,可以對於在傳輸資料的時候有一個依據來知道資料的可靠程度。
近年大數據的數量飛速成長,而龐大的數據會產生大量的應用。每一個應用都會產生資料交換,XML (eXtensible Markup Language)作為現今通用的網路資料交換格式,隨著網際網路資料的增長,也已經同樣具有大數據(Big Data)的特徵。本研究建構XML真實度模型應用程式開發介面(XML Veracity Model API)來解決大數據在資料傳輸中真實度不易量化的問題。XML文件真實度基於資料理解性有很多面向可以做詮釋,使用真實度模型API,使用者可以自行設計自己所認為的文件真實度的維度以及屬性,並產生量化的分數。且為了因應現今串流資料應用的增加,且又因串流XML資料又有結構上的問題,難以驗證真實度。本研究將真實度模型應用到串流資料,並且使用Apache Spark 增加模型的處理效能,來達到快速處理串流XML的目的,以及驗證真實度模型的設計架構。\\
\vspace{1.8em}
\noindent\textbf{關鍵字}:巨量資料、XML、串流、Apache Spark、真實度
\end{abstract}

\end{center}
\newpage