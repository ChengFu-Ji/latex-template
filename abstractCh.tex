\addcontentsline{toc}{section}{摘要}
\begin{center}

\large{基於Apache Spark 建構串流 XML Veracity 真實度之模型}\\
\vspace{0.83cm}

\makebox[3cm][r]{\large{學生:}}
\makebox[3cm][l]{\large{紀承甫}}
\hfill
\makebox[3cm][r]{\large{指導教授:}}
\makebox[3cm][l]{\large{陳世穎}}
\vspace{0.83cm}
\\
\large{國立臺中科技大學}\large{ 資訊工程系}
\vspace{0.83cm}
\renewcommand{\abstractname}{摘要}
\begin{abstract}
近年大數據的數量飛速成長,而龐大的數據會產生大量的應用。每一個應用都會產生資料交換,XML (eXtensible Markup Language)作為現今通用的網路資料交換格式,隨著網際網路資料的增長,也已經同樣具有大數據(Big Data)的特徵。本研究建構XML真實度模型應用程式開發介面(XML Veracity Model API)來解決大數據在資料傳輸中真實度不易量化的問題。XML文件真實度基於資料理解性有很多面向可以做詮釋,使用真實度模型API,使用者可以自行設計自己所認為的文件真實度的維度以及屬性,並產生量化的分數。且為了因應現今串流資料應用的增加,且又因串流XML資料又有結構上的問題,難以驗證真實度。本研究將真實度模型應用到串流資料,並且使用Apache Spark 增加模型的處理效能,來達到快速處理串流XML的目的,以及驗證真實度模型的設計架構。\\
\vspace{1.8em}
\noindent\textbf{關鍵字}:巨量資料、XML、串流、Apache Spark、真實度
\end{abstract}

\end{center}
\newpage