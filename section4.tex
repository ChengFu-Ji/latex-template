\section{Veracity模型API}
本研究使用OOP的觀念與特性在Apache Spark上面實做真實度模型,在前面有提到對於資料的真實度評價可以從很多面向來做探討,換句話說,可以讓使用者決定所謂的真實度需要包含哪一些屬性,還有真實度的量化方法、量化公式等,藉由OOP的方式,可以讓Veracity真實度模型的詮釋方式更有彈性,量化方式可以隨著不同的應用做變化,搭配Apache Spark 可以加速整個對於大型XML文件處理的效能。
\subsection{物件導向程式設計}
物件導向程式設計(Object-Oriented Programming,簡稱OOP)是一種抽象化的程式開發方法,
\subsection{模型建立}
Veracity真實度模型的建立,其中最重要的就是彈性,也就是用物件到像程式設計的特性來達到
\subsection{模型實作}