\section{結論與未來工作}

XML在大數據的應用下做為資料交換的格式有其重要性,而資料交換的時候如何確保資料的可信度以及真實度會是一個很重要的議題。\\\par
本研究提出了基於資料理解性的真實度模型API,提出一個方式解決真實度不易表達也不易量化的問題。並且使用物件導向的觀念來設計模型,提高模型的程式重用以及程式設計彈性。並且實作範例模型放在Apache Spark叢集當中運行,以加速整體模型的處理效能。模型應用的串流XML文件,以解決串流XML資料因在串流當中結構的不確定性,而導致真實度驗證困難的問題。\\\par
本研究所提出的模型基於物件導向的設計理念,可以應用的不同的場域以及可以隨著資料特性的不同客製化出符合使用者資料特性的模型。這樣的特色可以有助於產學界降低模型的理解難度以及實作難度。\\\par
在實驗的部分,實驗中採用三種不同的資料來源,且部分資料採用自行撰寫的產生器依照資料的特性以及規則進行生成,用以驗證真實度模型之可用性。實驗數據中可以觀察到真實度模型可以將三種不同種類的資料根據基準文件成功分類出來,且符合基準文件真實度的被量測文件之真實度分數都明顯突出,驗證了模型的可用性。\\\par

本研究提出真實度模型與模型API,解決真實度不易表達也不易量化的問題,並且應用在串流XML。後續研究可以朝向平行化計算以及串流資料處理優化或是XML結構對於效能的議題上來做後續探討。